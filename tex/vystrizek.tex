\documentclass[12pt,a4paper]{report}

\newcommand{\Lets}{Let us }


\begin{document}

\textit{Genetic programming} (GP) is a technique inspired by biological evolution
that for a given problem tries to find computer programs able to solve that problem. 

A problem to be solved is given to GP in a form of \textit{fitness function}. 
Fitness function is a function which takes computer program as its input and 
returns numerical value called \textit{fitness} as output. 
The bigger fitness of a computer program is, the better solution of a problem.

GP maintains a collection of computer programs called \textit{population}. 
A member of population is called \textit{individual}. 
By running GP algorithm evolution of those individuals is performed.

Individuals are computer program \textit{expressions} kept as \textit{syntactic trees}. 
Basically those trees are rooted trees with a function symbol in each internal node 
and with constant symbol or variable symbol in each leaf node. 
Number of child nodes for each internal node corresponds to the number of arguments of a function whose
symbol is in that node.

Another crucial input besides fitness function is a collection of \textit{building blocks}.
It is collection of symbols (accompanied with an information about number of arguments).
Those symbols are used to construct trees representing individuals.  
\\\\
\Lets describe GP algorithm briefly:

At the beginning initial population is generated from building blocks randomly.

A step of GP algorithm is stochastic transformation of the current population into 	
the next population.

This step consists of two sub steps:
\begin{itemize} 
	\item Selection of \textit{parents} for individuals of the next population based on the fitness.
	      The bigger fitness of an individual of the current population is, 
	      the better chance of success being selected as parent it has.  
	\item Application of genetic operators (such as \textit{crossover}, 
	      \textit{reproduction} and \textit{mutation}) 
		  on parent individuals producing new individuals of the next population.  
\end{itemize}	  
This transformation is repeatedly applied for a predefined number of steps (which is called 
number of \textit{generations}) or until some predefined criterion is met.	
\\\\
\Lets now look on GP at more detail. 	

\end{document}