
\documentclass[conference]{IEEEtran}
%\usepackage{stmaryrd}
\usepackage{amsfonts}


%\usepackage{graphicx,times,psfig,amsmath} % Add all your packages here
%\usepackage{hyperref}
\usepackage{amsmath}
%\usepackage{amssymb}
\usepackage[utf8]{inputenc}
\usepackage{qtree}
\usepackage{xspace}


\newcommand{\Lets}{Let us\xspace}
\newcommand{\lets}{let us\xspace}
\newcommand{\lterm}{$\lambda$-term\xspace}
\newcommand{\lterms}{$\lambda$-terms\xspace}
\newcommand{\lhead}{$\lambda$-head\xspace}
\newcommand{\lheads}{$\lambda$-heads\xspace}
\newcommand{\la}{\leftarrow\xspace}
\newcommand{\Lp}  {\Lambda^{\prime}\xspace}
\newcommand{\tur}[3]{#1\vdash{}#2:#3}
\newcommand{\turst}[3]{$#1\vdash{}#2:#3$\xspace}
\newcommand{\GMS}{\turst{\Gamma}{M}{\sigma}}
\newcommand{\atTree}{@-tree\xspace}
\newcommand{\setDots}[2]{ \lbrace #1 , \dots , #2 \rbrace}
\newcommand{\lh}[1]{\lambda #1}
\newcommand{\sexprTree}{sexpr-tree\xspace}
\newcommand{\SexprTree}{Sexpr-tree\xspace}
\newcommand{\then}{\Rightarrow\xspace}
\newcommand{\lamb}[2]{( \lambda \, #1 \, . \, #2 )}
\newcommand{\lam}[2]{\lambda \, #1 \, . \, #2}
\newcommand{\ST}{\mathop{\mathrm{ST}}}
\newcommand{\FV}{\mathop{\mathrm{FV}}}
\newcommand{\Scomb }{\mathbf{S}}
\newcommand{\Kcomb }{\mathbf{K}}
\newcommand{\Icomb }{\mathbf{I}}
\newcommand{\bbarr}{\twoheadrightarrow_\beta}
\newcommand{\barr}{\rightarrow_\beta}
\newcommand{\beq}{=_\beta}
\newcommand{\eearr}{\twoheadrightarrow_\eta}
\newcommand{\earr}{\rightarrow_\eta}
\newcommand{\eeq}{=_\eta}
\newcommand{\bearr}{\rightarrow_{\beta\eta}}
\newcommand{\bbeearr}{\twoheadrightarrow_{\beta\eta}}
\newcommand{\beeq}{=_{\beta\eta}}
\newcommand{\etar}{\twoheadrightarrow_\eta}
\newcommand{\ered}{$\eta$-reduction\xspace}
\newcommand{\bnf}{$\beta$-\textit{nf}\xspace}
\newcommand{\enf}{$\eta$-\textit{nf}\xspace}
\newcommand{\eenf}{$\eta^{-1}$-\textit{nf}\xspace}
\newcommand{\beenf}{$\beta\eta^{-1}$-\textit{nf}\xspace}
\newcommand{\benf}{$\beta\eta$-\textit{nf}\xspace}
\newcommand{\bredex}{$\beta$-redex\xspace} 
\newcommand{\lnf}{\textit{lnf}\xspace}
\newcommand{\Ae}{\mathop{\mathrm{\AE}}}
\newcommand{\Bcomb }{\mathbf{B}}   
\newcommand{\BBcomb }{\mathbf{B*}}
\newcommand{\Ccomb }{\mathbf{C}}   
\newcommand{\CCcomb }{\mathbf{C'}}
\newcommand{\SScomb }{\mathbf{S'}}
\newcommand{\ar}{\rightarrow\xspace}
\newcommand{\T}{\mathbb{T}\xspace}
\newcommand{\Real}{\mathbb{R}}


\hyphenation{op-tical net-works semi-conduc-tor IEEEtran}

\IEEEoverridecommandlockouts    % to create the author's affliation portion

\textwidth 178mm    % <------ These are the adjustments we made 10/18/2005
\textheight 239mm   % You may or may not need to adjust these numbers again
\oddsidemargin -7mm
\evensidemargin -7mm
\topmargin -6mm
\columnsep 5mm

\begin{document}


% paper title: Must keep \ \\ \LARGE\bf in it to leave enough margin.
\title{\ \\ \LARGE\bf Typed Functional Genetic Programming}

\author{Tom\'{a}\v{s} K\v{r}en}


\maketitle

\begin{abstract}
In this paper, generalization of the standard genetic programming (GP)
for simply typed lambda calculus is presented. We use population 
initialization method parameterized by simple search strategy. 
First described strategy corresponds to standard ramped half-and-half method, 
second one corresponds to exhaustive systematic search and third one is a 
novel geometric strategy, which outperforms standard method in success 
rate, time consumption and average individual size in two experiments. 
Other performance enhancements based on theory of lambda calculus are 
proposed and supported by experiment. Abstraction elimination is 
utilized to enable use of simple tree-swapping crossover.
\end{abstract}
% no key words

\section{Introduction}


\PARstart{O}{ur} approach aims to play with the full arsenal given by simply typed lambda calculus, thus we begin our reasoning with an exhaustive systematic search
in mind. Our second goal is to construct a system generalizing Standard 
GP \cite{koza92}.
In order to satisfy both these goals the designed system should be 
parameterized by some simple piece of code that makes the difference
between exhaustive systematic search and standard but 
quiet arbitrary ramped half-and-half generating method. 

Those two design goals also differentiate our system from 
the three state of the art systems for typed GP known to us.

~\\
\textit{[Kurzívou jsou neformální poznámky]\\\\
\textbf{Yu} - (články: evenParity, polyGP co sem ted našel, ten o burzách co mám v kindlu)
Odlišnosti:
(1) - generování se nedělá systematicky: pokud strom dojde do místa
kde funkci nemá dát jaký parametr, tak místo tý funkce dá nějaký 
terminál.(uvádí 85\% uspěšnost)  
(2) - Ty křížení ma trochu jinak než v tom článku o even parity,
musim zjistit jak má udělaný že se jí nedostane např prom \#3 někam,
kde neni definovaná když dělá přesun podstromu, v tom starym to ale 
bylo myslim založený na tom, že nedovolovala vnější proměnný uvnitř
lambda termu - což platí i nadále. čili vtom to určitě nehraje full deck.
Naopak se víc zaměřuje na polymorfizmus a další věci.\\
\textbf{kanadani}\\ 
- o dost silnější typovej systém\\
- System F, i jim dynamicky vznikaj typy\\
- moc silný typoví systém, takže generování už je dost
  složitý, to se otiskuje v silně nestandardním algoritmu\\
\textbf{kombinátoři} \\
- vůbec nepoužívaj Lambda Abstrakce\\
- univerzální genetickej operator\\
- taky řešej systematický prohledávání\\
\textbf{A eště tak nak všeobecně}\\
- geometrická strategie generování termů jakožto hybrid systematického a náhodného sytylu --- to že jsme si vytičili ty dva cíle který plníme tou 
jednoduchou strategií, tak máme možnost nalízt jednoduchou strategii která 
je na pomezí obou cílů - taková strategie je právě ta naše geometrická 
a ukazuje se že se bohulibě chová právě k strašáku GP -- \textbf{bloatu}.\\
- teoretické LC konstrukty s výhodou použity - @-stromy a eta-redukce\\
- křížení by eliminace 
}


 

\section{Generating method}

\textit{
- Unifished term je term který může obsahovat jako list uzel tvaru (tau,Gamma), takzvaný unfinished leaf (UL).\\
- Pracujeme s prioritní frontou rozdělaných lambda termů.\\ 
- Priorita je dána počtem UL v termu.\\
- Na počátku je do fronty vložen jediný UT = (tau,gama)\\
- Z fronty vyndám term s nejnižším \#UL, pokud má \#UL = 0, pak je to vygenerovaný jedinec, pokud ne tak vezmu jeho DFS-první UL a expanduju ho. Výsledkem expandování je několik (possibly 0) unfin. termů, které jsou až na expandovaný uzel totožné s původním UT, daný UT je nahrazen novým pod-UT. \\
- Expandování probíhá následovně: \\
(a) je to (a-b,Gama) pak tento list nahradim novým podstromem x.(b;Gama,x:a)\\
(b) je to (alfa,Gama) pak pro každý f:(T1-T2-...-Tn-alpha) el Gama 
nahradíme tento list novým podstromem ( f (T1,Gama) (T2,Gama) .. (Tn,Gama) )
\\\\
- Prohledávací strategie má pak funkci filtru následníků. \\
- Pokud se fronta vyprázdní ještě před vygenerováním požadovaného 
  počtu termů tak zbytek dogenerujeme stejným zpusobem, tzn do 
  fronty přihodíme (tau,gama)  \\
- Systematická strategie nezahodí nic (Pro ní generování 
  kolapsuje do A* algoritmu).\\
- Ramped-half-and-half zahodí vždy vše krom jediného.\\
- Geometrická strategie nezahodí s pravděpodobností q na hloubka UL \\
}

\textbf{$\eta$-normalization ...}

\textit{- pač je to generovany v lnf, neboli beta-eta na -1 nf kde 
eta na -1 je eta expanze tak je chytrý transformovat 
to do beta-eta nf kde  
}

\section{Crossover}

Design goal behind our approach to crossover operation is
to try to generalize standard tree swapping crossover.
The crossover operation in standard GP is performed 
by swapping randomly selected subtrees in each parent 
S-expression.

For typed lambda terms two difficulties arise: Types and variables.
We will show how to crossover typed \lterm trees in both 
\atTree and \sexprTree notation.
As in standard GP our crossover will be performed by swapping
two subtrees. But now with constraint that both subtrees have
the same type.

Variables bring more difficulties then types do.
This problem arises from variables that are free in subterms corresponding 
to swapped subtrees. 

Following example illustrates the problem. 
\Lets have these two parent trees with selected nodes in bold.\\

\Tree [.$\lh{x_1}$ [.f [.$\lh{x_2}$ [.\textbf{g} $x_2$ c ] ] $x_1$ ] ]
\Tree [.$\lh{x_1}$ [.h $x_1$ $\mathbf{x_1}$ ] ]

~\\The swap of subtrees results in following trees:\\

\Tree [.$\lh{x_1}$ [.f [.$\lh{x_2}$ $\mathbf{x_1}$ ] $x_1$ ] ]
\Tree [.$\lh{x_1}$ [.h $x_1$ [.\textbf{g} $\mathbf{x_2}$ \textbf{c} ] ] ]

~\\The problem is that variable $x_2$ in the second tree
is not bound by any $\lambda$-head and since
it is not element of $\Gamma$, the second tree is not well-typed \lterm.  

We can avoid dealing with this problem by avoiding use of variables.
This can be achieved by process called abstraction elimination. 

\subsection{Abstraction elimination}

\textit{Abstraction elimination} is a process of transforming 
an arbitrary \lterm into \lterm that contains no lambda abstractions
and no bound variables.
The newly produced \lterm may contain function applications, 
free symbols from former \lterm and some new symbols standing for 
combinators $\Scomb$, $\Kcomb$ and $\Icomb$. \\

Those combinators are defined as:
\begin{align*}
\Scomb &= \lam{f\,g\,x}{f\,x\,(g\,x)} \\
\Kcomb &= \lam{x\,y}{x} \\
\Icomb &= \lam{x}{x} 
\end{align*}


\Lets describe transformation $\Ae$ performing this process.
\begin{align*}
\Ae[x]           &= x &\\[0.4em]
\Ae[\,(M\,N)\,]  &= (\Ae[M]\;\;\Ae[N]) &\\[0.4em]
\Ae[\lam{x}{x}]  &= \Icomb &\\
\Ae[\lam{x}{M}]  &= (\Kcomb~\Ae[M]) &\textbf{if } x \not\in \FV(M)\\
\Ae[\lam{x}{\lamb{y}{M}}] &= \Ae[\lam{x}{\Ae[\lam{y}{M}]}]  
&\textbf{if } x \in \FV(M)\\
\Ae[\lam{x}{(M\,N)}] &= (\Scomb~\Ae[\lam{x}{M}]~\Ae[\lam{x}{N}])  
&\textbf{if } x \in \FV(M) \\
&&\vee x \in \FV(N)
\end{align*}


This is simple version of this process. More optimized version,
in the means of the size of resulting term and its performance
is following one, presented in \cite{jones87}.

This version operates with more combinators:
\begin{align*}
\Bcomb  &= \lam{f\,g\,x}{f\,(g\,x)} \\
\Ccomb  &= \lam{f\,g\,x}{f\,x\,g} \\
\SScomb &= \lam{c\,f\,g\,x}{c\,(f\,x)\,(g\,x)} \\
\BBcomb &= \lam{c\,f\,g\,x}{c\,(f\,(g\,x))} \\
\CCcomb &= \lam{c\,f\,g\,x}{c\,(f\,x)\,g} 
\end{align*}

And the transformation can be written as follows.
\begin{align*}
\Ae[x]           &= x &\\
\Ae[\,(M\,N)\,]  &= (\Ae[M]\;\;\Ae[N]) &\\
\Ae[\lam{x}{M}]  &= A[ x ; \Ae[M] ] &
\\[1em]
A[x;x]           &= \Icomb &\\
A[x;y]           &= (\Kcomb~y)\\
A[x;\,(M\,N)\,]  &= Opt[~\Scomb~(A[x;M])~(A[x;N])~]
\\[1em]
Opt[~\Scomb~(\Kcomb~M)~(\Kcomb~N)~]   &= \Kcomb~(M~N)\\
Opt[~\Scomb~(\Kcomb~M)~\Icomb~]       &= M\\
Opt[~\Scomb~(\Kcomb~M)~(\Bcomb~N~L)~] &= \BBcomb~M~N~L\\
Opt[~\Scomb~(\Kcomb~M)~N~]            &= \Bcomb~M~N\\
Opt[~\Scomb~(\Bcomb~M~N)~(\Kcomb~L)~] &= \CCcomb~M~N~L\\
Opt[~\Scomb~M~(\Kcomb~N)~]            &= \Ccomb~M~N\\
Opt[~\Scomb~(\Bcomb~M~N)~L~]          &= \SScomb~M~N~L
\end{align*}

As is stated in \cite{jones87},
the biggest disadvantage of this technique is that the translated
term is often much larger than in its lambda form --- the size of
the translated term can be proportional to the
square of the size of the original term. 

But the advantage is also tempting --- no need to deal with variables
and lambda heads.

\subsection{Typed subtree swapping}
\label{typed-swapping}

First thing to do in standard subtree swapping is to select random node
in the first parent. 

We modify this procedure so that we allow
selection only of those nodes with such a type that there exists  
a node in the second parent with the same type.

Standard subtree swapping crossover as a first thing selects 
whether the selected node will be inner node (usually with probability 
$p_{ip} = 90\%$) or leaf node (with probability 10\%).

We are in a more complicated situation, because one of those 
sets may be empty, because of allowing only nodes with possible "partner"
in the second parent. Thus we do this step only if both sets are
nonempty. 

After selecting a node in the first parent we select node in the
second parent such that type of that node must by the same as the type 
of the first node. Again, this may eliminate the "90-10" step of
first deciding whether the selected node will be internal node 
or leaf node.

When both nodes are selected we may swap the trees. \\

If the abstraction elimination was performed, then 
since the trees are of the same type and there are no variables to be 
moved from their scope, the offspring trees are well typed.\\


Both \sexprTree and \atTree are able to be crossed by this 
mechanism. But \atTree has more possibilities then \atTree.
This comes from the fact that every subtree of the \sexprTree
corresponds to a subtree of \atTree, but there are subtrees
of \atTree that do not correspond to a subtree of a \sexprTree.\\

Following example should clarify this.

\Tree[.@	
   [.@ \textbf{f} x ]
   [.y ]  		 			
]
\Tree[.\textbf{f} x y ]~\\

In \atTree, \textbf{f} is leaf thus subtree, 
whereas in \sexprTree it is internal node thus not a subtree.\\ 

Another nice property of \sexprTree{}s with no lambdas 
is that they are the same representation as S-expressions
used by standard GP.\\

Again, similarly as for standard version, 
a maximum permissible depth $D_{created}$ 
for offspring individuals is defined (e.g. $D_{created} = 17$).
If one of the offspring has greater depth than this limit, then 
this offspring is replaced by the first parent in the result of 
the crossover operator. If both offspring exceeds this limit, than 
both are replaced by both parents.  

For \atTree the $D_{created}$ must be larger since
\atTree (without lambdas) is a binary tree. This 
enlargement is approximately proportionate to average number of 
function arguments. We use generous $D_{created} = 17\times3$. 



\section{Experiments}

\subsection{Simple Symbolic Regression}

\textit{Simple Symbolic Regression} is a problem described
in \cite{koza92}. Objective of this problem is to 
find a function $f(x)$ that fits a sample
of twenty given points. The target function is 
function $f_{t}(x) = x^4 + x^3 + x^2 + x$.  

Desired type of generated programs $\sigma$ and 
building blocks context $\Gamma$ are following.
\begin{align*}
\sigma = \Real \ar &\Real\\
\Gamma = \{
  (+)  &: \Real \ar \Real \ar \Real    ,\\
  (-)  &: \Real \ar \Real \ar \Real    ,\\
  (*)  &: \Real \ar \Real \ar \Real    ,\\
  rdiv &: \Real \ar \Real \ar \Real    ,\\
  sin  &: \Real \ar \Real              ,\\
  cos  &: \Real \ar \Real              ,\\
  exp  &: \Real \ar \Real              ,\\ 
  rlog &: \Real \ar \Real              \}
\end{align*}
where
\begin{align*}
rdiv(p,q) &= \begin{cases} 1 &\mbox{if } q = 0 \\
p/q & \mbox{otherwise } \end{cases}  \\
rlog(x) &= \begin{cases} 0 &\mbox{if } x = 0 \\
log(\vert x\vert) & \mbox{otherwise}. \end{cases}
\end{align*}

Fitness function is computed as follows

$$ fitness(f) =  \sum\limits_{i=1}^{20}{ \vert f(x_i)-y_i }\vert   $$

where $(x_i,y_i)$ are 20 data samples from $[-1,1]$, such that $y_i = f_t(x_i)$.\\

An individual $f$ such that $\vert f(x_i)-y_i \vert < 0.01 $ for all data samples is 
considered as a correct individual.

\subsection{Artificial Ant}
\subsection{Even-parity problem}


\section{Conclusions}

Our goal was to design and implement a system
performing GP over some typed functional programming language.
As this typed functional programming language we have chosen the 
\textit{simply typed lambda calculus}.
We have also obligated ourselves to do it in a such way that 
generalizes the standard GP, rather than
crates completely new system.

We may say that we have chosen the simplest possible
option in both areas --- the simply typed calculus and the standard GP.
We see this decision as fortunate one --- since it enables us add more
complex features after sufficient exploration of those simplest cases.
   
We have started the journey toward this goal by study 
of subject that intersects both those areas:
Method which for desired type and context generates lambda terms.  

The way to obtain this method started with inference rules 
(the defining concepts of the typed lambda calculus),
it proceeded through term generating grammars and finished 
with inhabitation trees.

A* algorithm in combination with inhabitation trees
has been utilized to drive systematic enumeration
of typed \lterms in their \lnf. This enumeration was further 
parameterized by simple search strategy in order to enable
such different approaches as \textit{systematic} generation 
and \textit{ramped half-and-half} generation to be definable
in the means of simple search strategy.\\

With this apparatus in hands we introduced novel approach
to term generation by defining the \textit{geometric} search 
strategy. This strategy is further parameterized by parameter
$q$,  but since we wanted to avoid suspicion that success
of this generating method depends on fine-tuning of this
parameter, we used its default value $q = 0.75$ in every experiment.

After being generated all terms undergo process of \textit{abstraction
elimination}, which enables our simple tree swapping crossover operation
which is only genetic operator that we have used in all experiments.

We examined it in three different experiments, which all 
supported the idea that it has very desirable qualities.

First two experiments, \textit{Simple Symbolic Regression} and 
\textit{Artificial Ant}, were performed in order to compare
the \textit{geometric} strategy with the standard GP term 
generating method \textit{ramped half-and-half}. 

In both experiments were observed improvements in the following aspects.

\begin{enumerate}
 \item Success rate was improved.
 \item Minimal number of individuals needed to be processed in order to yield 
       correct solution with probability 99\% was lowered.
 \item Run time was significantly reduced. 
 \item Average size of a term was decreased.
\end{enumerate}

In the case of Artificial Ant problem, all improvements were significant.

These results make me believe that \textit{geometric} strategy might
be welcomed reinforcements in the fight against the bogey of 
the GP community --- the \textit{bloat}.
Or at least, it seems to work for those two experiments in a such way.\\


In the third experiment, the \textit{even-parity} problem, geometric strategy 
showed ability to yield correct solution in the initial generation
8 times out of 200 runs (throughout 5 experiments), which is interesting
result since it is uninformed search. 

Five consecutive experiments with even-parity problem supported 
hypothesis that $\eta$-normalization of generated terms enhances performance.
We have also seen that use \atTree{}s instead of more traditional \sexprTree{}s
as tree representation of individuals is able to enhanced performance.

Use of optimized abstraction elimination instead of its basic variant 
showed significant improvements in time consumption with little or 
non effects on performance rates.

Implemented system was designed with importance of interactivity in mind,
resulting in server/client architecture for core/GUI components.


\begin{thebibliography}{22}


\bibitem{koza92}
  John R. Koza,
  \emph{Genetic Programming: On the Programming of Computers by Means of Natural Selection}.
  MIT Press, Cambridge, MA,
  1992. 

\bibitem{koza05}
  Koza, J.R., Keane, M., Streeter, M., Mydlowec, W.,Yu, J., Lanza, G. 
  \emph{Genetic Programming IV: Routine Human-Competitive Machine Intelligence.} 
  Springer, 2005. ISBN 978-0-387-26417-2 

\bibitem{fg}
 Riccardo Poli, William B. Langdon, Nicholas F. McPhee
 \emph{A Field Guide to Genetic Programming}.
 Lulu Enterprises, UK Ltd, 2008.

\bibitem{yu01}
  T. Yu. 
  \emph{Hierachical processing for evolving recursive and modular 
        programs using higher order functions and lambda abstractions}. 
  Genetic Programming and Evolvable Machines,
  2(4):345–380, December 2001. ISSN 1389-2576.


\bibitem{montana95}
D. J. Montana. 
\emph{Strongly typed genetic programming.} 
Evolutionary Computation, 3(2): 199–230, 1995.
%URL \url{ http://vishnu.bbn.com/papers/stgp.pdf }. nefacha

\bibitem{haynes96}
T. D. Haynes, D. A. Schoenefeld, and R. L. Wainwright. 
\emph{Type inheritance in strongly typed genetic programming.} 
In P. J. Angeline and K. E. Kinnear, Jr., editors, Advances
in Genetic Programming 2, chapter 18, pages 359–376.
MIT Press, Cambridge, MA, USA, 1996. ISBN 0-262-01158-1. 
%URL \url{http://www.mcs.utulsa.edu/~rogerw/papers/Haynes-hier.pdf}.

\bibitem{olsson94}
J. R. Olsson. 
\emph{Inductive functional programming using incremental program 
transformation and Execution of logic programs by 
iterative-deepening A* SLD-tree search.} 
Dr scient thesis, University of Oslo, Norway, 1994.

\bibitem{kes}
Forrest Briggs, Melissa O’Neill.
\emph{Functional Genetic Programming and Exhaustive
Program Search with Combinator Expressions.}
International Journal of Knowledge-based and Intelligent Engineering Systems,
Volume 12 Issue 1, Pages 47-68, January 2008. 


\bibitem{barendregt84}
H. P. Barendregt,
\emph{The Lambda Calculus: its Syntax and Semantics}, 
revised ed., North-Holland, 1984.

\bibitem{barendregt92}
H. Barendregt , S. Abramsky , D. M. Gabbay , T. S. E. Maibaum.
\emph{Lambda Calculi with Types.} 
Handbook of Logic in Computer Science, 1992. 

\bibitem{barendregt10}

  Henk Barendregt, Wil Dekkers, Richard Statman,
  \emph{Lambda Calculus With Types}.
  Cambridge University Press,
  2010. 
  %URL \url{http://www.cs.ru.nl/~henk/book.pdf}.

\bibitem{jones87}
Simon Peyton Jones. 
\emph{The Implementation of Functional Programming Languages}. 
Prentice Hall, 1987.


\bibitem{AIAMA}
	Stuart J. Russell, Peter Norvig,
	\emph{Artificial Intelligence: A Modern Approach}.
	Pearson Education,
	2003. 


\end{thebibliography}

% that's all folks
\end{document}
