
\documentclass[conference]{IEEEtran}
%\usepackage{stmaryrd}
\usepackage{amsfonts}


%\usepackage{graphicx,times,psfig,amsmath} % Add all your packages here
%\usepackage{hyperref}
%\usepackage{amsmath}
%\usepackage{amssymb}
\usepackage[utf8]{inputenc}

\hyphenation{op-tical net-works semi-conduc-tor IEEEtran}

\IEEEoverridecommandlockouts    % to create the author's affliation portion

\textwidth 178mm    % <------ These are the adjustments we made 10/18/2005
\textheight 239mm   % You may or may not need to adjust these numbers again
\oddsidemargin -7mm
\evensidemargin -7mm
\topmargin -6mm
\columnsep 5mm

\begin{document}


% paper title: Must keep \ \\ \LARGE\bf in it to leave enough margin.
\title{\ \\ \LARGE\bf Typed Functional Genetic Programming}

\author{Tom\'{a}\v{s} K\v{r}en}


\maketitle

\begin{abstract}
In this paper, generalization of the standard genetic programming (GP)
for simply typed lambda calculus is presented. We use population 
initialization method parameterized by simple search strategy. 
First described strategy corresponds to standard ramped half-and-half method, 
second one corresponds to exhaustive systematic search and third one is a 
novel geometric strategy, which outperforms standard method in success 
rate, time consumption and average individual size in two experiments. 
Other performance enhancements based on theory of lambda calculus are 
proposed and supported by experiment. Abstraction elimination is 
utilized to enable use of simple tree-swapping crossover.
\end{abstract}
% no key words

\section{Introduction}


\PARstart{O}{ur} approach aims to play with the full arsenal given by simply typed lambda calculus, thus we begin our reasoning with an exhaustive systematic search
in mind. Our second goal is to construct a system generalizing Standard 
GP \cite{koza92}.
In order to satisfy both these goals the designed system should be 
parameterized by some simple piece of code that makes the difference
between exhaustive systematic search and standard but 
quiet arbitrary ramped half-and-half generating method. 

Those two design goals also differentiate our system from 
the three state of the art systems for typed GP known to us.

~\\
\textit{[Kurzívou jsou neformální poznámky]\\\\
\textbf{Yu} - (články: evenParity, polyGP co sem ted našel, ten o burzách co mám v kindlu)
Odlišnosti:
(1) - generování se nedělá systematicky: pokud strom dojde do místa
kde funkci nemá dát jaký parametr, tak místo tý funkce dá nějaký 
terminál.(uvádí 85\% uspěšnost)  
(2) - Ty křížení ma trochu jinak než v tom článku o even parity,
musim zjistit jak má udělaný že se jí nedostane např prom \#3 někam,
kde neni definovaná když dělá přesun podstromu, v tom starym to ale 
bylo myslim založený na tom, že nedovolovala vnější proměnný uvnitř
lambda termu - což platí i nadále. čili vtom to určitě nehraje full deck.
Naopak se víc zaměřuje na polymorfizmus a další věci.\\
\textbf{kanadani}\\ 
- o dost silnější typovej systém\\
- System F, i jim dynamicky vznikaj typy\\
- moc silný typoví systém, takže generování už je dost
  složitý, to se otiskuje v silně nestandardním algoritmu\\
\textbf{kombinátoři} \\
- vůbec nepoužívaj Lambda Abstrakce\\
- univerzální genetickej operator\\
- taky řešej systematický prohledávání\\
\textbf{A eště tak nak všeobecně}\\
- geometrická strategie generování termů jakožto hybrid systematického a náhodného sytylu --- to že jsme si vytičili ty dva cíle který plníme tou 
jednoduchou strategií, tak máme možnost nalízt jednoduchou strategii která 
je na pomezí obou cílů - taková strategie je právě ta naše geometrická 
a ukazuje se že se bohulibě chová právě k strašáku GP -- \textbf{bloatu}.\\
- teoretické LC konstrukty s výhodou použity - @-stromy a eta-redukce\\
- křížení by eliminace 
}


 

\section{Generating method}

\textit{
- Unifished term je term který může obsahovat jako list uzel tvaru (tau,Gamma), takzvaný unfinished leaf (UL).\\
- Pracujeme s prioritní frontou rozdělaných lambda termů.\\ 
- Priorita je dána počtem UL v termu.\\
- Na počátku je do fronty vložen jediný UT = (tau,gama)\\
- Z fronty vyndám term s nejnižším \#UL, pokud má \#UL = 0, pak je to vygenerovaný jedinec, pokud ne tak vezmu jeho DFS-první UL a expanduju ho. Výsledkem expandování je několik (possibly 0) unfin. termů, které jsou až na expandovaný uzel totožné s původním UT, daný UT je nahrazen novým pod-UT. \\
- Expandování probíhá následovně: \\
(a) je to (a-b,Gama) pak tento list nahradim novým podstromem x.(b;Gama,x:a)\\
(b) je to (alfa,Gama) pak pro každý f:(T1-T2-...-Tn-alpha) el Gama 
nahradíme tento list novým podstromem ( f (T1,Gama) (T2,Gama) .. (Tn,Gama) )
\\\\
- Prohledávací strategie má pak funkci filtru následníků. \\
- Pokud se fronta vyprázdní ještě před vygenerováním požadovaného 
  počtu termů tak zbytek dogenerujeme stejným zpusobem, tzn do 
  fronty přihodíme (tau,gama)  \\
- Systematická strategie nezahodí nic (Pro ní generování 
  kolapsuje do A* algoritmu).\\
- Ramped-half-and-half zahodí vždy vše krom jediného.\\
- Geometrická strategie nezahodí s pravděpodobností q na hloubka UL \\
}

\textbf{$\eta$-normalization ...}

\textit{- pač je to generovany v lnf, neboli beta-eta na -1 nf kde 
eta na -1 je eta expanze tak je chytrý transformovat 
to do beta-eta nf kde  
}

\section{Crossover}
\textit{- design goal: proper generalization of standard tree 
swapping crossover\\ 
- … výceméně zkopčit z diplomky\\
\textbf{@-tree representation ...}\\
...
}
\section{Experiments}

\subsection{Simple Symbolic Regression}
\subsection{Artificial Ant}
\subsection{Even-parity problem}


\section{Conclusions}


\begin{thebibliography}{22}


\bibitem{koza92}
  John R. Koza,
  \emph{Genetic Programming: On the Programming of Computers by Means of Natural Selection}.
  MIT Press, Cambridge, MA,
  1992. 

\bibitem{koza05}
  Koza, J.R., Keane, M., Streeter, M., Mydlowec, W.,Yu, J., Lanza, G. 
  \emph{Genetic Programming IV: Routine Human-Competitive Machine Intelligence.} 
  Springer, 2005. ISBN 978-0-387-26417-2 

\bibitem{fg}
 Riccardo Poli, William B. Langdon, Nicholas F. McPhee
 \emph{A Field Guide to Genetic Programming}.
 Lulu Enterprises, UK Ltd, 2008.

\bibitem{yu01}
  T. Yu. 
  \emph{Hierachical processing for evolving recursive and modular 
        programs using higher order functions and lambda abstractions}. 
  Genetic Programming and Evolvable Machines,
  2(4):345–380, December 2001. ISSN 1389-2576.


\bibitem{montana95}
D. J. Montana. 
\emph{Strongly typed genetic programming.} 
Evolutionary Computation, 3(2): 199–230, 1995.
%URL \url{ http://vishnu.bbn.com/papers/stgp.pdf }. nefacha

\bibitem{haynes96}
T. D. Haynes, D. A. Schoenefeld, and R. L. Wainwright. 
\emph{Type inheritance in strongly typed genetic programming.} 
In P. J. Angeline and K. E. Kinnear, Jr., editors, Advances
in Genetic Programming 2, chapter 18, pages 359–376.
MIT Press, Cambridge, MA, USA, 1996. ISBN 0-262-01158-1. 
%URL \url{http://www.mcs.utulsa.edu/~rogerw/papers/Haynes-hier.pdf}.

\bibitem{olsson94}
J. R. Olsson. 
\emph{Inductive functional programming using incremental program 
transformation and Execution of logic programs by 
iterative-deepening A* SLD-tree search.} 
Dr scient thesis, University of Oslo, Norway, 1994.

\bibitem{kes}
Forrest Briggs, Melissa O’Neill.
\emph{Functional Genetic Programming and Exhaustive
Program Search with Combinator Expressions.}
International Journal of Knowledge-based and Intelligent Engineering Systems,
Volume 12 Issue 1, Pages 47-68, January 2008. 


\bibitem{barendregt84}
H. P. Barendregt,
\emph{The Lambda Calculus: its Syntax and Semantics}, 
revised ed., North-Holland, 1984.

\bibitem{barendregt92}
H. Barendregt , S. Abramsky , D. M. Gabbay , T. S. E. Maibaum.
\emph{Lambda Calculi with Types.} 
Handbook of Logic in Computer Science, 1992. 

\bibitem{barendregt10}

  Henk Barendregt, Wil Dekkers, Richard Statman,
  \emph{Lambda Calculus With Types}.
  Cambridge University Press,
  2010. 
  %URL \url{http://www.cs.ru.nl/~henk/book.pdf}.

\bibitem{jones87}
Simon Peyton Jones. 
\emph{The Implementation of Functional Programming Languages}. 
Prentice Hall, 1987.


\bibitem{AIAMA}
	Stuart J. Russell, Peter Norvig,
	\emph{Artificial Intelligence: A Modern Approach}.
	Pearson Education,
	2003. 


\end{thebibliography}

% that's all folks
\end{document}
