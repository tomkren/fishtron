\documentclass[12pt,a4paper]{report}

\setlength\textwidth{145mm}
\setlength\textheight{247mm}
\setlength\oddsidemargin{15mm}
\setlength\evensidemargin{15mm}
\setlength\topmargin{0mm}
\setlength\headsep{0mm}
\setlength\headheight{0mm}


\usepackage[utf8]{inputenc}

\usepackage{qtree}

\usepackage{amsmath}
\usepackage{amsfonts}
\usepackage{amsthm}


\usepackage{graphicx}

\usepackage{hyperref}

%% Balíček hyperref, kterým jdou vyrábět klikací odkazy v PDF,
%% ale hlavně ho používáme k uložení metadat do PDF (včetně obsahu).
%% POZOR, nezapomeňte vyplnit jméno práce a autora.
%\usepackage[ps2pdf,unicode]{hyperref}   % Musí být za všemi ostatními balíčky
%\hypersetup{pdftitle=Typed Functional Genetic Programming}
%\hypersetup{pdfauthor=Tomáš Křen}

% Tato makra přesvědčují mírně ošklivým trikem LaTeX, aby hlavičky kapitol
% sázel příčetněji a nevynechával nad nimi spoustu místa. Směle ignorujte.
\makeatletter
\def\@makechapterhead#1{
  {\parindent \z@ \raggedright \normalfont
   \Huge\bfseries \thechapter. #1
   \par\nobreak
   \vskip 20\p@
}}
\def\@makeschapterhead#1{
  {\parindent \z@ \raggedright \normalfont
   \Huge\bfseries #1
   \par\nobreak
   \vskip 20\p@
}}
\makeatother


\newcommand\Vtextvisiblespace[1][.3em]{%
  \mbox{\kern.06em\vrule height.3ex}%
  \vbox{\hrule width#1}%
  \hbox{\vrule height.3ex}}

\title{Typed Functional Genetic Programming}
\author{Tomáš Křen}
\date{Prague 2013}

\begin{document}

% Trochu volnější nastavení dělení slov, než je default.
\lefthyphenmin=2
\righthyphenmin=2

%%% Titulní strana práce

\pagestyle{empty}
\begin{center}

\large

Charles University in Prague

\medskip

Faculty of Mathematics and Physics

\vfill

{\bf\Large MASTER THESIS}

\vfill

%%% \centerline{\mbox{\includegraphics[width=60mm]{../img/logo.eps}}}

%\begin{figure}[!ht]
%  \centering
%  \includegraphics{logo.eps}
%  \caption{Default}\label{fig:default}
%\end{figure}

%%%%%%%%%%%%%%%%%%%%%%%%%%%%%%%%%%%%%%%%%%%%%%%%%%%%%%%%%%%%%%%%%%%%%%%%%%%%%%%
% Stačí todle odkomentovat a dát nahoře LaTeX (F2) , pak DVI->PDF (F9)  a pak View PDF
%\includegraphics[scale=0.5]{logo.eps}
\includegraphics[scale=0.15]{logomff.png}

\vfill
\vspace{5mm}

{\LARGE Tomáš Křen}

\vspace{15mm}

% Název práce přesně podle zadání
{\LARGE\bfseries Typed Functional Genetic Programming}

\vfill

% Název katedry nebo ústavu, kde byla práce oficiálně zadána
% (dle Organizační struktury MFF UK)
%%%%Name of the department or institute
%Department of Theoretical Computer Science and Mathematical Logic\\
%{\small Department of Theoretical Computer Science and Mathematical Logic} \\
{\fontsize{0.46cm}{1em}\selectfont 
Department of Theoretical Computer Science and Mathematical Logic}

\vfill

\begin{tabular}{rl}

Supervisor of the master thesis: & RNDr. Petr Pudlák, Ph.D. \\
\noalign{\vspace{2mm}}
Study programme: & Theoretical Computer Science \\ %Teoretická informatika \\
\noalign{\vspace{2mm}}
Specialization: & 
%Neprocedurální programování a umělá inteligence \\
{\fontsize{0.3cm}{1em}\selectfont 
%Neprocedurální programování a umělá inteligence} \\
Non-Procedural Programming and Artificial Intelligence} \\
\end{tabular}

\vfill

% Zde doplňte rok
Prague 2013

\end{center}

\newpage

%%% Následuje vevázaný list -- kopie podepsaného "Zadání diplomové práce".
%%% Toto zadání NENÍ součástí elektronické verze práce, nescanovat.

%%% Na tomto místě mohou být napsána případná poděkování (vedoucímu práce,
%%% konzultantovi, tomu, kdo zapůjčil software, literaturu apod.)

%% on tam měl %% \openright

\noindent
Dedication.

\newpage

%%% Strana s čestným prohlášením k diplomové práci

\vglue 0pt plus 1fill

\noindent
I declare that I carried out this master thesis independently, and only with the cited
sources, literature and other professional sources.

\medskip\noindent
I understand that my work relates to the rights and obligations under the Act No.
121/2000 Coll., the Copyright Act, as amended, in particular the fact that the Charles
University in Prague has the right to conclude a license agreement on the use of this
work as a school work pursuant to Section 60 paragraph 1 of the Copyright Act.

\vspace{10mm}

\hbox{\hbox to 0.5\hsize{%
In ........ date ............
\hss}\hbox to 0.5\hsize{%
signature of the author
\hss}}

\vspace{20mm}
\newpage


%%% Povinná informační strana diplomové práce

\vbox to 0.5\vsize{
\setlength\parindent{0mm}
\setlength\parskip{5mm}

Název práce:
Název práce
% přesně dle zadání

Autor:
Jméno a příjmení autora

Katedra:  % Případně Ústav:
Název katedry či ústavu, kde byla práce oficiálně zadána
% dle Organizační struktury MFF UK

Vedoucí diplomové práce:
Jméno a příjmení s tituly, pracoviště
% dle Organizační struktury MFF UK, případně plný název pracoviště mimo MFF UK

Abstrakt:
% abstrakt v rozsahu 80-200 slov; nejedná se však o opis zadání diplomové práce

Klíčová slova:
% 3 až 5 klíčových slov

\vss}\nobreak\vbox to 0.49\vsize{
\setlength\parindent{0mm}
\setlength\parskip{5mm}

Title:
% přesný překlad názvu práce v angličtině

Author:
Jméno a příjmení autora

Department:
Název katedry či ústavu, kde byla práce oficiálně zadána
% dle Organizační struktury MFF UK v angličtině

Supervisor:
Jméno a příjmení s tituly, pracoviště
% dle Organizační struktury MFF UK, případně plný název pracoviště
% mimo MFF UK v angličtině

Abstract:
% abstrakt v rozsahu 80-200 slov v angličtině; nejedná se však o překlad
% zadání diplomové práce

Keywords:
% 3 až 5 klíčových slov v angličtině

\vss}

\newpage


\tableofcontents	
	
%\chapter{Introduction}
\chapter*{Introduction}
\addcontentsline{toc}{chapter}{Introduction}
	
\chapter{Definitions}
	
	Let us first say some basic definitions.
		
\section{Lambda term}
	
	Let $V$ be set of {\it variable names}.  \\* 
	Let $C$ be set of {\it constant names}.	 \\*		
	Then $\Lambda$ is set of {\it $\lambda$-terms} inductively defined as follows:
	
	\begin{align*}
		 x   \in V                 &\Rightarrow     x   \in \Lambda \\
		 c   \in C                 &\Rightarrow     c   \in \Lambda \\
		 M,N \in \Lambda           &\Rightarrow ( M N ) \in \Lambda \\
		 x   \in V , M \in \Lambda &\Rightarrow ( \lambda x . M ) \in \Lambda 
	\end{align*} 
	
\section{Type}

	Let $A$ be set of {\it atomic type names}. \\*
	Then $\mathbb{T}$ is set of {\it types} inductively defined as follows:
	
	\begin{align*}
	\alpha      \in A          &\Rightarrow     \alpha   \in \mathbb{T} \\
	\sigma,\tau \in \mathbb{T} &\Rightarrow ( \sigma \rightarrow  \tau ) \in \mathbb{T} 
	\end{align*} 
	
	
\section{Statement of a form $M : \sigma$}

	Let $\Lambda$ be set of {\it $\lambda$-terms}. \\*
	Let $\mathbb{T}$ be set of {\it types}.       \\*
	A {\it statement} $M : \sigma$ is a pair $(M,\sigma) \in \Lambda \times \mathbb{T}$. \\*
	$M : \sigma$ is vocalized as {\it "$M$ has type $\sigma$"}.\footnote{ 
	$M : \sigma$ can be also imagined as $M \in \sigma$ } \\*
	The type $\sigma$ is the {\it predicate} and the term $M$ is the
	{\it subject} of the statement.  
	
\section{Context}

	Let $\Gamma \in \mathfrak P \left({\Lambda \times  \mathbb{T}}\right)$. 
	($\Gamma$ is a set of {\it statements} of a form $M : \sigma$.)	\\*
	Then $\Gamma$ is {\it context} if it obeys following 
	conditions\footnote{
	The $\pi_1$ corresponds to the projection of the first component of the Cartesian product.
	}:
	\begin{align*}
		 \forall (x,\sigma) \in \Gamma &: x \in V \cup C \\
		 \forall s_1,s_2 \in \Gamma &: s_1 \neq s_2 \Rightarrow \pi_1(s_1) \neq \pi_1(s_2)
    \end{align*}
    
	In other words context is a set of statements with distinct variables or constants as subjects.
	
	
\section{Statement of a form $\Gamma \vdash M : \sigma$}

	By writing $\Gamma \vdash M : \sigma$ we say 
	{\it statement $M : \sigma$ is derivable from context $\Gamma$ }.

	We construct valid statements of form $\Gamma \vdash M : \sigma$ by using inference rules.
	
		
\section{Inference rule}		
	
	Basically speaking, inference rules are used for deriving statements of a form 
	$\Gamma \vdash M : \sigma$ from yet derived statements of such a form.
	Those inference rules are written in the following form:
	
	\begin{equation*}
		\frac{\Gamma_1 \vdash M_1 : \sigma_1 \qquad
			  \Gamma_2 \vdash M_2 : \sigma_2 \quad
			  \dotsm \quad
		      \Gamma_n \vdash M_n : \sigma_n}
		     {\Gamma_{n+1} \vdash M_{n+1} : \sigma_{n+1}}
	\end{equation*}	
	
	Suppose we have yet derived statements 
	$\Gamma_1 \vdash M_1 : \sigma_1 ,
	 \Gamma_2 \vdash M_2 : \sigma_2 ,
	 \dots ,
	 \Gamma_n \vdash M_n : \sigma_n$. 
	It allows as to use the inference rule to derive statement
	\mbox{ $\Gamma_{n+1} \vdash M_{n+1} : \sigma_{n+1}$ }.
	 
	For deriving statements including types of a form 
	$(\sigma \rightarrow \tau)$ are essential those two 
	inference rules:
	
	\begin{equation*}
		\frac{\Gamma \vdash M : \sigma \rightarrow \tau \qquad
			  \Gamma \vdash N : \sigma }
		     {\Gamma \vdash (M N) : \tau }
	\end{equation*}	
	
	\begin{equation*}
		\frac{\Gamma \cup \{ ( x,\sigma ) \} \vdash M : \tau }
		     {\Gamma \vdash (\lambda x . M) : \sigma \rightarrow \tau }
	\end{equation*}		 
	 
	This kind of inference rules allows us to derive new statements from yet derived statements, but 
	what if we do not have any statement yet? 
	For this purpose we have other kinds of inference rules such as {\it axiom} inference rule:   
	
	\begin{equation*}
		\frac{( x , \sigma )  \in \Gamma}
		     {\Gamma \vdash x : \sigma}
	\end{equation*}	
	
	Let us consider an example statement of a form $\Gamma \vdash M : \sigma$ :
	
	\[
		\{\} \vdash (\lambda f . (\lambda x . (f x) )) : 
		(\sigma \rightarrow \tau) \rightarrow ( \sigma \rightarrow \tau ) 
	\]
		
	This statement is derived as follows: 
	
	\begin{equation*}
	\dfrac{
		\dfrac{ (f,\sigma \rightarrow \tau) \in \{ (f,\sigma \rightarrow \tau) , (x,\sigma)  \}  }
		     { \{ (f,\sigma \rightarrow \tau) , (x,\sigma)  \} \vdash f : \sigma \rightarrow \tau }
		\qquad
		\dfrac{ (x,\sigma) \in \{ (f,\sigma \rightarrow \tau) , (x,\sigma)  \}  }
		     { \{ (f,\sigma \rightarrow \tau) , (x,\sigma)  \} \vdash x : \sigma }
		 }
		 {
			\dfrac{		 	
		 		\{ (f,\sigma \rightarrow \tau) , (x,\sigma)  \} \vdash (f x) : \tau
		 	}{
				\dfrac{\{ (f,\sigma \rightarrow \tau) \} \vdash (\lambda x . (f x) ) : 
				\sigma \rightarrow \tau}
				{ \{ \} \vdash (\lambda f . (\lambda x . (f x) ) ) 
				  : (\sigma \rightarrow \tau) \rightarrow (\sigma \rightarrow \tau) }
		 	}
		 }
	\end{equation*}		
	
	\section{Term generating grammar}
	Inference rules are good for deriving statements of a form $\Gamma \vdash M : \sigma$, but our
	goal is slightly different; we would like to generate many $\lambda$-terms M for a given type 
	$\sigma$ and context $\Gamma$.
	
	Our approach will be to take each inference rule and transform it to a rule of term generating
	grammar. With this term generating grammar it will be much easier to reason about generating 
	$\lambda$-terms.
	
It won't be a grammar in classical sense because we will be operating with infinite sets of
nonterminal symbols and rules. \footnote{TODO : mention terminal symbols - situation around 
variables and their construction with ' symbol.}

Let $Non = Type \times Context $ be our {\it nonterminal} set. 
So for every $i \in Non$ is $i = (\sigma_i , \Gamma_i )$.

Let's consider each relevant inference rule and its corresponding grammar rule.

First inference rule is {\it implication elimination} also known as 
{\it modus ponens}: 
\[
	\frac{\Gamma \vdash M : \sigma \rightarrow \tau \qquad
		  \Gamma \vdash N : \sigma }
	     {\Gamma \vdash (M N) : \tau }
\]
\\
For every $\sigma, \tau \in \mathbb{T}$ and for every {\it context} 
$\Gamma \in \mathfrak P \left({\Lambda \times  \mathbb{T}}\right)$ there is a grammar rule of a form\footnote{ 
Terminal symbols for parenthesis and normally {\it space} now \textvisiblespace \quad (for {\it function application} operator) are visually highlighted. }: 
\[	
	( \tau , \Gamma )  \longmapsto
	\bigg( ( \sigma \rightarrow \tau , \Gamma ) 
	  \mbox{ \Vtextvisiblespace[1em] } ( \sigma , \Gamma ) \bigg)
\]
\\

Second inference rule is {\it implication introduction}: 
\[
	\frac{\Gamma \cup \{ ( x,\sigma ) \} \vdash M : \tau }
	     {\Gamma \vdash (\lambda x . M) : \sigma \rightarrow \tau }
\]
\\
$\forall \sigma, \tau \in \mathbb{T}$ 
$\forall${\it context} $\Gamma \in \mathfrak P \left({\Lambda \times  \mathbb{T}}\right) $ 
$\forall x \in V $ such that there is no $(x,\rho) \in \Gamma$ 
there is a grammar rule:
\[ 
	( \sigma \rightarrow \tau , \Gamma )  \longmapsto
	\bigg( \mbox{ {\Large $\lambda$ x . }}( \tau , \Gamma \cup \{ (x,\sigma) \} ) \quad \bigg)
\]
\\	

Third inference rule is {\it axiom}: 
\[
		\frac{( x , \sigma )  \in \Gamma}
		     {\Gamma \vdash x : \sigma}
\]
\\
$\forall \sigma \in \mathbb{T}$ 
$\forall${\it context} $\Gamma \in \mathfrak P \left({\Lambda \times  \mathbb{T}}\right) $ 
$\forall x \in V \cup C $ such that $(x,\sigma) \in \Gamma$ 
there is a grammar rule:
\[ 
	( \sigma , \Gamma )  \longmapsto \mbox{ {\Large x}}
\]
\\

We will demonstrate $\lambda$-term generation on example. 
Again on $(\lambda f . (\lambda x . (f x) ))$. 
We would like to generate $\lambda$-term of a type 
$(\sigma \rightarrow \tau) \rightarrow (\sigma \rightarrow \tau)$
with $\Gamma = \{\}$.
\begin{align*}
	& ((\sigma \rightarrow \tau) \rightarrow (\sigma \rightarrow \tau),\{\}) \\ 
	\longmapsto & \Big( \mbox{ {\Large $\lambda$f.}}
	  ( \sigma \rightarrow \tau , \{ (f,\sigma \rightarrow \tau) \} ) 
	~ \Big)
	\\
	\longmapsto & 
	\Big( \mbox{ {\Large $\lambda$f. }}
		\Big( \mbox{ {\Large $\lambda$x. }}
	  	 	( \tau , \{ (f,\sigma \rightarrow \tau) , (x,\sigma) \} ) 
		~ \Big)  	 
	~ \Big)
	\\
	\longmapsto & 
	\Big( \mbox{ {\Large $\lambda$f. }}
		\Big( \mbox{ {\Large $\lambda$x. }}	  	 	
	  	 	\Big( 
	  	 	  ( \sigma \rightarrow \tau , \{ (f,\sigma \rightarrow \tau) , (x,\sigma) \} ) 
			  \mbox{ \Vtextvisiblespace[1em] } 
			  ( \sigma , \{ (f,\sigma \rightarrow \tau) , (x,\sigma) \} )  \Big) 
		~ \Big)  	 
	 ~ \Big)
	\\
	\longmapsto & 
	\Big( \mbox{ {\Large $\lambda$f. }}
		\Big( \mbox{ {\Large $\lambda$x. }}	  	 	
	  	 	\Big( 
	  	 	  \mbox{ {\Large f}} 
			  \mbox{ \Vtextvisiblespace[1em] } 
			  ( \sigma , \{ (f,\sigma \rightarrow \tau) , (x,\sigma) \} ) \Big) 
		~ \Big)  	 
	~ \Big)		
	\\
	\longmapsto & 
	\Big( \mbox{ {\Large $\lambda$f. }}
		\Big( \mbox{ {\Large $\lambda$x. }}	  	 	
	  	 	\Big( 
	  	 	  \mbox{ {\Large f}} 
			  \mbox{ \Vtextvisiblespace[1em] } 
			  \mbox{{\Large x}} \Big) 
		~ \Big)  	 
	~ \Big)
\end{align*}

\subsection{"Barendregt-like" inference and grammar rules}
\label{barlike}

Inference rule 1: 
\[
	\frac{\Gamma \cup \{ (x_1,\tau_1),\dots,(x_n,\tau_n) \} \vdash M : \alpha }
	     {\Gamma \vdash (\lambda x_1 \dots x_n . M) : 
	     \tau_1 \rightarrow \dots \rightarrow \tau_n \rightarrow \alpha }
\]
\\
... there is a grammar rule:
\[ 
	( \tau_1 \rightarrow \dots \rightarrow \tau_n \rightarrow \alpha , \Gamma )  \longmapsto
	\bigg( \mbox{ {\Large 
	$\lambda x_1 \dots x_n .$ 
	}}( \alpha , \Gamma \cup \{ (x_1,\tau_1),\dots,(x_n,\tau_n) \} ) \quad \bigg)
\]
\\

Inference rule 2: 
\[
	\frac{ (f , \tau_1 \rightarrow \dots \rightarrow \tau_n \rightarrow \alpha ) \in \Gamma \qquad
	       \Gamma \vdash M_1 : \tau_1 \quad
	       \dotsm \quad
	       \Gamma \vdash M_n : \tau_n        
	      }
	     {\Gamma \vdash (f M_1 \dots M_n) : \alpha}
\]
\\
... there is a grammar rule:
\[ 
	( \alpha , \Gamma )  \longmapsto
	\bigg( \mbox{ {\Large f }}
	  \mbox{ \Vtextvisiblespace[1em] } 
	  ( \tau_1 , \Gamma )
	  \mbox{ \Vtextvisiblespace[1em] } 
	  \dots
	  \mbox{ \Vtextvisiblespace[1em] } 
	  ( \tau_n , \Gamma )
	  \quad \bigg)
\]
\\	



\section{Inhabitation tree}

Now we will introduce \textit{Inhabitation tree}, structure slightly different from
\textit{Inhabitation machine}, which was introduced in \cite{barendregt10} by Henk Barendregt.
We can think about Inhabitation tree as about unfolded Inhabitation machine.
\\\\
\textbf{TODO}: TALK ABOUT And-Or-graph \\
\textbf{TODO}: TALK ABOUT "Barendregt-like" subsection \ref{barlike}
\\\\
Let's see some example: 

\begin{align*}
\sigma =& ~ \mathbb{B} \rightarrow  \mathbb{B} \rightarrow  \mathbb{B} \\ 
\Gamma =& \{ ~ true : \mathbb{B}  \\
        &  , ~ nand :  \mathbb{B} \rightarrow \mathbb{B} \rightarrow \mathbb{B} ~ \}
\end{align*}


\Tree
[.\text{ $\mathbb{B} \rightarrow \mathbb{B} \rightarrow \mathbb{B}$ }  
	[.\textbf{$\lambda$x_1 x_2 } 
		[.\text{ $\mathbb{B}$ } 
			\textbf{true}  
			[.\textbf{nand} 
				\qroof{ ~~ $\dotsb$ ~~ }.\text{ $\mathbb{B}$ }
				\qroof{ ~~ $\dotsb$ ~~ }.\text{ $\mathbb{B}$ } 
			]
			\textbf{x_1}
			\textbf{x_2}
		]
	]
]

		
	
		\subsection{Inhabitation Machine}
	\section{Roadmap}
	\section{Conversion to SKI combinators}

	\section{Genetic Programming}
		\subsection{Term generating}
		\subsection{Crossover}
		\subsection{Mutation}

\chapter{ Designed system }	
	\section{Top level view}
		\subsection{Comments about main source files}
			\subsubsection{ Eva.hs }
	\section{Term generating}
		\subsection{A* algorithm}
	\section{Crossover}
		\subsection{Finding same types}
		\subsection{Two basic options}
		Resolve problems with free variables or avoid variables completely. 
		
	\section{Mutation}
		\subsection{Using term generation}


\chapter{Problems}
	In this section will be presented usage of the system in order to solve specific problems.
		
	
		\section{Even Parity Problem}
		\section{Big Context}
		\section{Fly}
		\section{Simple Symbolic Regression}
		\section{Artificial Ant}
		\section{Boolean Alternate}
	

%\chapter{Conclusion}
\chapter*{Conclusion}
\addcontentsline{toc}{chapter}{Conclusion}	
	

\begin{thebibliography}{9}

\bibitem{barendregt10}
  Henk Barendregt, Wil Dekkers, Richard Statman,
  \emph{Lambda Calculus With Types}.
  Cambridge University Press,
  2010. \\
  \url{http://www.cs.ru.nl/~henk/book.pdf}

\end{thebibliography}

	
	
\end{document}
