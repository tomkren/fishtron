\documentclass[12pt]{report}

\usepackage{amsmath}
\usepackage{amsfonts}
\usepackage[utf8]{inputenc}

\title{Typed Functional Genetic Programming}
\author{Tomáš Křen}
\date{April 2013}

\begin{document}
	\maketitle
	
	
\chapter{Introduction}

	
\chapter{Definitions}
	
	Let us first say some basic definitions.
		
	\section{Lambda term}
		
		Let $V$ be set of {\it variable names}.  \\* 
		Let $C$ be set of {\it constant names}.	 \\*		
		Then $\Lambda$ is set of {\it $\lambda$-terms} inductively defined as follows:
		
		\begin{align*}
			 x   \in V                 &\Rightarrow     x   \in \Lambda \\
			 c   \in C                 &\Rightarrow     c   \in \Lambda \\
			 M,N \in \Lambda           &\Rightarrow ( M N ) \in \Lambda \\
			 x   \in V , M \in \Lambda &\Rightarrow ( \lambda x . M ) \in \Lambda 
		\end{align*} 
	
	\section{Type}
	
		Let $A$ be set of {\it atomic types}. \\*
		Then $\mathbb{T}$ is set of {\it types} inductively defined as follows:
		
		\begin{align*}
		\alpha      \in A          &\Rightarrow     \alpha   \in \mathbb{T} \\
		\sigma,\tau \in \mathbb{T} &\Rightarrow ( \sigma \rightarrow  \tau ) \in \mathbb{T} 
		\end{align*} 
	
	
	\section{Statement with :}
	Let $\Lambda$ be set of {\it $\lambda$-terms}. \\*
	Let $\mathbb{T}$ be set of {\it types}.       \\*
	A {\it statement} is a pair $(M,\sigma) \in \Lambda \times \mathbb{T}$. \\*
	Notation $M : \sigma$. \\* 
	The type $\sigma$ is the {\it predicate} and the term $M$ is the
	{\it subject} of the statement.  
	
	\section{Context}
	Let $\Gamma$ be a set of {\it statements}.	\\*
	Then $\Gamma$ is {\it context} if it obeys following conditions:
	\begin{align*}
		 \forall (x,\sigma) \in \Gamma &: x \in V \cup C \\
		 \forall s_1,s_2 \in \Gamma &: s_1 \neq s_2 \Rightarrow fst(s_1) \neq fst(s_2)
    \end{align*}
    
	In other words context is a set of statements with distinct variables as subjects.
	
	
	\section{Statement with $\vdash$}
	By writing $ \Gamma \vdash M : \sigma$ we say 
	{\it statement $M : \sigma$ is derivable from context $\Gamma$ }.
	
		
	\section{Inference rule}	
	We construct valid {\it statements with $\vdash$} by using inference rules.
	
	
		
	
	\section{Term generating grammar}

	\section{Inhabitation tree}
		\subsection{Inhabitation Machine}
	\section{Roadmap}
	\section{Conversion to SKI combinators}

	\section{Genetic Programming}
		\subsection{Term generating}
		\subsection{Crossover}
		\subsection{Mutation}

\chapter{ Designed system }	
	\section{Top level view}
	\section{Term generating}
		\subsection{A* algorithm}
	\section{Crossover}
		\subsection{Finding same types}
		\subsection{Two basic options}
		Resolve problems with free variables or avoid variables completely. 
		
	\section{Mutation}
		\subsection{Using term generation}


\chapter{Problems}
	In this section will be presented usage of the system in order to solve specific problems.
		
	
		\section{Even Parity Problem}
		\section{Big Context}
		\section{Fly}
		\section{Simple Symbolic Regression}
		\section{Artificial Ant}
		\section{Boolean Alternate}
	

\chapter{Conclusion}
	
	
\chapter{Sandbox ..}
	
	...	
	
	% Example formula
  	
  	\begin{align}
    	E &= mc^2                              \\
    	m &= \frac{m_0}{\sqrt{1-\frac{v^2}{c^2}}}
  	\end{align}
	
	
\end{document}